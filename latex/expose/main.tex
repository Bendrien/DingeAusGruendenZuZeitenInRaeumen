
\documentclass{article}

% character encoding
\usepackage[utf8]{inputenc}
\usepackage[ngerman]{babel}
\usepackage[T1]{fontenc}

\usepackage[onehalfspacing]{setspace}
%\usepackage[doublespacing]{setspace}

\usepackage{enumerate}
\usepackage{amsmath}
\usepackage{graphicx}
\usepackage{url}
\usepackage{siunitx}
\sisetup{
  locale = DE ,
  per-mode = symbol
}

\begin{document}


\title{
    \includegraphics[width=0.5\textwidth]{fig/audiokommtu-01.png}\\
    Vergleich des Spotify MIR Algorithmus mit Probandenbefragungen\\
    \vspace{2mm}
    \small{Exposé für Empirisch-wissenschaftliches Arbeiten - Gruppe 11}
}
\author
{
Bendrien, Anyere \\
382004\\
\small\texttt{anyere.bendrien@posteo.de}
\and
Wagenbach, Maximilian \\
382110\\
\small\texttt{maxijuli@t-online.de}
\and
Cycon, Sebastian \\
382126\\
\small\texttt{cycon@campus.tu-berlin.de}
}

\maketitle

\thispagestyle{empty} \newpage \setcounter{page}{1}


\section{Exposé}

Im Rahmen dieser Forschungsarbeit beschäftigen wir uns mit dem Thema des Kategorisierens von Musikstücken und des automatischen Wiedererkennens von musikalischen Elementen innerhalb eines Stückes.
Der dafür etablierte Fachbegriff ist das sogenannte Music Information Retrieval (kurz: MIR).
Dies ermöglicht es aus Musikstücken grundlegende Metriken wie zum Beispiel das Tempo, die Tonart, das Tongeschlecht oder die Geschwindigkeit zu extrahieren \cite{Casey2008}.
Aufbauend auf diesen lassen sich subjektive Empfindungen, wie zum Beispiel die Fröhlichkeit, die Tanzbarkeit oder die Komplexität eines Musikstücks extrapolieren \cite{Sturm2013}.
Ein aktuell schwerwiegendes Problem im Feld des MIR ist die Bewertung und der Vergleich verschiedener MIR Algorithmen \cite{Downie2004} \cite{Urbano_2013}. Wir haben den Ansatz gewählt die Ergebnisse eines MIR Algorithmus mit Bewertungen von Probanden zu vergleichen.


Als Datensatz haben wir "`10 Songs für die einsame Insel"' gewählt.
Bei dieser Studie wurden 50 Nutzer dazu befragt welche 10 Lieder sie auf eine einsame Insel mitnehmen würden.
Anschließend sollten sie die von ihnen gewählten Lieder nach subjektiven Kriterien, wie zum Beispiel Intensität, Erregung oder Traurigkeit, bewerten.
Der Musikstreaming-Anbieter "`Spotify"' ermöglicht den Zugriff auf seine Datenbank an MIR Daten.
Wir wollen betrachten wie die Bewertung von Musikstücken durch Menschen mit der automatischen Bewertung durch einen Algorithmus korreliert.


Das Thema erhält seine Relevanz, weil es für die durch Techniken des MIR erzeugten Informationen zahlreiche Verwendungsmöglichkeiten gibt. 
Es ist zum Beispiel denkbar automatisch Playlisten für Nutzer von Musiksoftware zu erstellen oder dem Nutzer Musik vorzuschlagen, die er noch nicht besitzt, welche aber seinem Musikgeschmack entspricht.
Auch für die Anbieter von großen Musikdatenbanken sind MIR Algorithmen ein wichtiges Gebiet, weil sie es ermöglichen die riesigen Datenmengen zu kategorisieren und damit maschinell durchsuchbar zu machen.

\newpage

\bibliography{literatur}
\bibliographystyle{ieeetr}

\end{document}