\section*{Einleitung}
\label{sec:Einleitung}

Im Digitalen Zeitalter ist es immer wichtiger große Datenmengen zu untersuchen und speziell an Informationen über deren Inhalt zu gelangen.
Mit Hilfe verschiedener Techniken ist es möglich, vielfältige Informationen über den Inhalt von digitalen Daten zu extrahieren, wodurch sie sich für eine informationstechnische Weiterverarbeitung eigenen.
Die Entwicklung solcher Methoden für musikalische Inhalte allgemein ist Gegenstand des interdisziplinären Forschungsgebiets \textit{Music Information Retrieval}, kurz \textit{MIR}.
So ist es beispielsweise möglich, mit computergestützten Verfahren, grundlegende Metriken wie das Tempo, die Tonart oder das Tongeschlecht aus einem Musikstück zu extrahieren\,\cite{Casey2008}.
Aufbauend auf diesen einfachen Metriken lassen sich komplexere Attribute, welche subjektive Empfindungen entsprechen, wie zum Beispiel die Fröhlichkeit, die Tanzbarkeit oder die Komplexität eines Musikstücks, vorhersagen \cite{Sturm2013}.
Die in diesem Kontext gewonnenen Informationen werden auch \textit{Audio Features} genannt.
Unter anderem lassen sich diese MIR Audio Features in Beziehung zueinander stellen, um so Zusammenhänge zwischen den zugrunde liegenden Daten zu ermitteln.
So ist es beispielsweise möglich automatisch Playlisten für einen Nutzer zu erstellen oder dem Nutzer Musik vorzuschlagen, die er noch nicht besitzt, welche aber seinem Musikgeschmack entspricht.

%Das Streamen von medialen Inhalten über das Internet ist im gesellschaftlichen Alltag kaum noch wegzudenken. % kürzen?
Der On-Demand Streaming-Dienst \textit{Spotify}\footnote{Website: \url{https://www.spotify.com/}} hat sich auf Audiodaten spezialisiert.
Er stellt seinen Nutzern eine umfangreiche Sammlung von Musik zum Hören über das Internet zur Verfügung und wendet auf diesen Daten eigene \textit{MIR} Verfahren an.
\textit{Spotify} extrahiert aus der Musik \textit{Audio Features} und macht diese über eine Schnittstelle öffentlich abrufbar.
Aufbauend darauf ist eine Vielzahl an informationstechnischer Weiterverarbeitungen möglich, für welche jedoch der Wahrheitsgehalt der abgerufenen Informationen einen großen Stellenwert hat.
Dieses Problem führt zu der schwer greifbaren und hochaktuellen Problematik im Feld des \textit{MIR}:
Wie lassen sich die Ergebnisse unterschiedlicher \textit{MIR} Verfahren und Algorithmen bewerten und mit einander vergleichen? \cite{Urbano_2013}

In der Praxis kommen verschiedene Evaluationsverfahren zum Einsatz.
Ein Verfahren mit Blick auf \textit{Gültigkeit}, \textit{Zuverlässigkeit} und \textit{Effizienz} \cite{Downie2004}, wird im Folgenden genauer beschrieben.
Die durch \textit{MIR} extrahierten \textit{Audio Features} werden einer Aggregation von Vergleichsinformationen gegenübergestellt und mit ihr abgeglichen (\textit{Gültigkeit}), um eine Aussage über ihren Wahrheitsgehalt treffen zu können.
Damit die aggregierten Daten als Vergleichsmaß herangezogen werden können, muss zuvor sichergestellt werden, dass sie sowohl inhaltlich als auch quantitativ einem Mindestmaß an Qualität entsprechen (\textit{Zuverlässigkeit}).
Ein solcher Datensatz wird auch als \textit{ground truth} bezeichnet. 

In dieser Studie werden die Audio Features von Spotify nach Gültigkeit geprüft.
Als gegenübergestellte Vergleichsinformation dienen subjektive Bewertungen von Musikstücken. Entnommen werden diese Daten aus der Studie "10 Insel..."
%Zum Vergleich werden resultierendaus der Studie "10en Informationen  Insel...." gegenübergestellt 
Aus dem Vergleich soll der Grad der Korrelation zwischen den Audio Features und den Vergleichsdaten hervorgehen und damit der Wahrheitsgehalt der Spotify Features beurteilt werden.
       


TO DO:
Ein bis zwei Beispiele aus bisherigen Studien 
Unsere Herangehensweise mit Folgerung zur Nullhypothese:
\textbf{Die von \textit{Spotify} bereitgestellten \textit{Audio Features} korrelieren mit einem Wert von ,5 oder mehr mit subjektiven Bewertungen von Menschen.}
alternative: \textbf{Die von \textit{Spotify} bereitgestellten \textit{Audio Features} korrelieren mit einem Korrelationskoeffizienten von 0,5 oder mehr mit den subjektiven Bewertungen von Menschen.}
