\section*{Einleitung}
\label{sec:Einleitung}
Aus digitalen Inhalten lassen sich vielfältige Informationen extrahieren, die für eine datentechnische Weiterverarbeitung geeignet sind.
Die Entwicklung solcher Methoden wird für musikalische Inhalte allgemein als \textit{Music Information Retrieval}, kurz \textit{MIR}, bezeichnet.
%interdisziplinäres Forschungsgebiet.
Beispielsweise ist es möglich, mit computergestützten Verfahren, grundlegende Metriken wie das Tempo, die Tonart oder das Tongeschlecht aus einem Musikstück zu extrahieren \cite{Casey2008}.
Aufbauend auf diesen lassen sich subjektive Empfindungen, wie zum Beispiel die Fröhlichkeit, die Tanzbarkeit oder die Komplexität eines Musikstücks vorhersagen \cite{Sturm2013}.
Die in diesem Kontext gewonnenen Informationen werden auch \textit{Audio Features} genannt.
Unter anderem lassen sich mit \textit{MIR} Zusammenhänge zwischen den zugrunde liegenden Daten ermitteln, um diese in Beziehung zueinander zu setzen.
So ist es beispielsweise möglich automatisch Playlisten für einen Nutzer zu erstellen oder dem Nutzer Musik vorzuschlagen, die er noch nicht besitzt, welche aber seinem Musikgeschmack entspricht. % kürzen?

%Das Streamen von medialen Inhalten über das Internet ist im gesellschaftlichen Alltag kaum noch wegzudenken. % kürzen?
Auf Audiodaten spezialisiert hat sich der On-Demand-Streaming-Dienst \textit{Spotify}\footnote{\url{https://www.spotify.com/}}.
Er stellt seinen Nutzern eine umfangreiche Sammlung von Musik zum Hören über das Internet zur Verfügung und wendet auf diesen Daten eigene \textit{MIR} Verfahren an.
\textit{Spotify} bietet eine Schnittstelle, die ihre aus der Musik extrahierten \textit{Audio Features} öffentlich abrufbar macht.\footnote{\url{https://developer.spotify.com/web-api/get-audio-features/}}
Dadurch ist eine Vielzahl an datentechnischer Weiterverarbeitung möglich, für welcher jedoch der Wahrheitsgehalt der abgerufenen Informationen einen großen Stellenwert hat.
Das führt zu einer aktuell schwer greifbaren Problematik im Feld des \textit{MIR}:
Die Bewertung und der Vergleich verschiedener \textit{MIR} Verfahren und Algorithmen.

In der Praxis kommen Evaluationsverfahren mit Blick auf \textit{Gültigkeit}, \textit{Zuverlässigkeit} und \textit{Effizienz} zum Einsatz \cite{Downie2004}, auf die im Folgenden genauer eingegangen wird.
Die durch \textit{MIR} extrahierten \textit{Audio Features} werden einer Aggregation von Vergleichsinformationen gegenübergestellt und mit ihr abgeglichen (\textit{Gültigkeit}), um eine Aussage über ihren Wahrheitsgehalt treffen zu können.
Damit die aggregierten Daten als Vergleichsmaß herangezogen werden können, muss zuvor sichergestellt werden, dass sie sowohl Inhaltlich als auch Quantitativ einem Mindestmaß an Qualität entsprechen (\textit{Zuverlässigkeit}).
Ein solcher Datensatz wird auch als \textit{ground truth} bezeichnet.


TO DO:
Ein bis zwei Beispiele aus bisherigen Studien 
\cite{Urbano_2013}.
Unsere Herangehensweise mit Folgerung zur Nullhypothese:
\textbf{Die von \textit{Spotify} bereitgestellten \textit{Audio Features} korrelieren mit einem Wert von ,5 oder mehr mit subjektiven Bewertungen von Menschen.}
