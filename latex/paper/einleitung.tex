\section*{Einleitung}
\label{sec:Einleitung}

Aus digitalen Inhalten lassen sich vielfältige Informationen extrahieren, die für eine datentechnische Weiterverarbeitung geeignet sind.
%So lassen sich beispielsweise Zusammenhänge zwischen den zugrunde liegenden Daten aufdecken, um diese in Beziehung setzen zu können.
Für musikalische Inhalte wird dieser Vorgang als \textit{Music Information Retrieval}, kurz \textit{MIR}, bezeichnet.
Es gibt Verfahren, die aus Musikstücken grundlegende Metriken wie das Tempo, die Tonart oder das Tongeschlecht extrahieren \cite{Casey2008}.
Aufbauend auf diesen lassen sich subjektive Empfindungen, wie zum Beispiel die Fröhlichkeit, die Tanzbarkeit oder die Komplexität eines Musikstücks vorhersagen \cite{Sturm2013}.
%Das Streamen von medialen Inhalten über das Internet ist im gesellschaftlichen Alltag kaum noch wegzudenken. % könnte gekürzt werden
Auf Audio spezialisiert hat sich der On-Demand-Streaming-Dienst \textit{Spotify}\footnote{\url{https://www.spotify.com}}.
Er stellt seinen Nutzern eine umfangreiche Sammlung von Musik zum Hören über das Internet zur Verfügung und wendet auf diesen Daten eigene \textit{MIR} Verfahren an.
So ist es ihm beispielsweise möglich automatisch Playlisten für die Nutzer zu erstellen oder dem Nutzer Musik vorzuschlagen, die er noch nicht besitzt, welche aber seinem Musikgeschmack entspricht.
